\documentclass[12pt,a4paper,brazilian, fleqn]{article}

\usepackage{babel}
\usepackage[utf8]{inputenc}
\usepackage[T1]{fontenc}
\usepackage{lmodern}

\usepackage{amssymb,amsfonts,amsmath}

\usepackage{tikz}
\usetikzlibrary{calc,intersections}

\usepackage{tcolorbox}
\tcbset{boxrule=0pt, top=0pt, bottom=0pt}

\DeclareMathOperator{\sen}{sen}

%https://tex.stackexchange.com/a/100406
%29.7 cm - 1cm - 1cm - 144.90/28.4 cm = 22.60 cm
\usepackage[a4paper, totalheight=22.60cm,includeheadfoot,left=1.5cm, right=1.0cm, top=1cm]{geometry}
\setlength{\headheight}{144.90pt}

\setkeys{Gin}{keepaspectratio}

\newcommand{\cabeca}{
    \begin{tikzpicture}
        \node(Logo) {\includegraphics[width=2.5cm]{logo.png}};
        % \node(Logo) {\includegraphics[width=4.1cm]{logo.png}};

        \node(Local) at (Logo.north east) [anchor=north west, yshift=-0.25cm,
            align=center, execute at begin node=\setlength{\baselineskip}{3ex}]
            {
                \huge{\textbf{Universidade Federal do Amazonas}} \\
                \large{\textbf{Instituto de Ciências Exatas e Tecnologia}} \\
                \large{\textbf{\Description}}
            };

        \node(Ident) at (Local.south west) [anchor=north west, yshift=-0.25cm,
            align=left, execute at begin node=\setlength{\baselineskip}{2em}]
            {
                Professor: {\fontfamily{augie}\selectfont \Professor} \\
                Aluno:
            };
        % \draw [thick] (Logo.south west) -- ($(Logo.south west -| Local.south east)$);
        % \draw [red] (Logo.north west) rectangle (Logo.south east);
        % \draw [blue] (Local.north west) rectangle (Local.south east);
        % \draw [green] (Ident.north west) rectangle (Ident.south east);
    \end{tikzpicture}
}

\usepackage{fancyhdr}
\fancyhead{}
\fancyfoot{}
\fancyhead[c]{\cabeca}
\fancyfoot[r]{\fontfamily{augie}\selectfont Boa sorte!}


\pagestyle{fancy}
\renewcommand{\headrulewidth}{0pt}
\renewcommand{\footrulewidth}{0pt}

\newcommand{\ratio}[1]{(#1\% da nota)}
%-----------------------------------CUT HERE-----------------------------------

\def\Description{Física Geral II -- Prova Final}
\def\Professor{Rodrigo de Farias Gomes}

\renewcommand{\vec}[1]{\overrightarrow{#1}}

\usepackage{siunitx}
\sisetup{locale = FR}

\begin{document}

\begin{tcolorbox}[colback=black!10, colframe=black!50, title=Observações]
    \begin{itemize}
        \item Todas as páginas com resposta devem ter o nome e matrícula do aluno
            escritos com caneta no início (cabeçalho) ou no final (rodapé). Páginas
            que não obedeçam a esse critério não serão usadas na avaliação
        \item As respostas devem ser escritas com caneta
    \end{itemize}
\end{tcolorbox}

\begin{enumerate}
    \item \ratio{33} Um pulso isolado, cuja forma de onda é dada por \(h(x + 5t)\), com \(x\) em centímetros e 
        \(t\) em segundos, é mostrado na figura abaixo para \(t = 0\). A escala do eixo vertical é definida 
        por \(h_s = 2\). (a) Qual é a velocidade e (b) qual o sentido de propagação do pulso? 
        (c) \textit{Esboce} \(h(x + 5t)\) em função de \(x\) para \(t = \SI{2}{s}\). (d) \textit{Esboce} \(h(x + 5t)\)
        em função de \(t\) para \(x = \SI{3}{cm}\).

        \begin{center}
        \begin{tikzpicture}[scale=2]
            \draw [thick] (0,0) -- (5,0);
            \draw [thick] (0,0) -- (0,2) node [midway, sloped, yshift=2em] {\(h(x)\)};
            \node [below left] at (0,0) {0};
            \node [left] at (0,2) {\(h_s\)};
            \foreach \x in {1,2,3,4,5} {
                \node [below] at (\x,0) {\x};
                \draw [black!40] (\x,0) |- (0,1);
                \draw [black!40] (\x,0) |- (0,2);
            }
            \draw [dashed,line width=2pt] (3,0) |- (0,2);
            \draw [line width=3pt] (0,0) -- (1,0) -- (3,2) -- (4,0) -- (5,0);
            \path (2,0) -- (3,0) node [midway, below, yshift=-1em] {\(x\)};
            \node [above right] at (4,1) {\(t=0\)};
        \end{tikzpicture}
        \end{center}

    \item \ratio{33} A equação de uma onda transversal que se propaga em uma corda muito longa é 
        \(y = \num{6.0} \sen{(\num{0.020}\pi x + \num{4.0}\pi t)}\), em que \(x\) e
        \(y\) estão em centímetros e \(t\) em segundos. Determine (a) a amplitude,
        (b) o comprimento de onda, (c) a frequência, (d) a velocidade, (e) o 
        sentido de propagação da onda e (f) a máxima velocidade transversal de uma 
        partícula da corda. (g) Qual é o deslocamento transversal em 
        \( x = \SI{3.5}{cm}\) para \(t = \SI{0,26}{s}\)?

    \item \ratio{34} Um pêndulo simples, com \SI{20}{cm} de comprimento e \SI{5.0}{g} de massa, está suspenso
        em um carro de corrida que se move a uma velocidade constante de \SI{70}{m/s}, descrevendo
        uma circunferência com \SI{50}{m} de raio. Se o pêndulo sofre pequenas oscilações na direção
        radial em torno da posição de equilíbrio, qual é a frequência das oscilações?
\end{enumerate}

\end{document}
