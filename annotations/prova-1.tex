\documentclass[12pt,a4paper,brazilian, fleqn]{article}

\usepackage{babel}
\usepackage[utf8]{inputenc}
\usepackage[T1]{fontenc}
\usepackage{lmodern}

\usepackage{amssymb,amsfonts,amsmath}

\usepackage{tikz}
\usetikzlibrary{calc,intersections}

\usepackage{tcolorbox}
\tcbset{boxrule=0pt, top=0pt, bottom=0pt}

\DeclareMathOperator{\sen}{sen}

%https://tex.stackexchange.com/a/100406
%29.7 cm - 1cm - 1cm - 144.90/28.4 cm = 22.60 cm
\usepackage[a4paper, totalheight=22.60cm,includeheadfoot,left=1.5cm, right=1.0cm, top=1cm]{geometry}
\setlength{\headheight}{144.90pt}

\setkeys{Gin}{keepaspectratio}

\newcommand{\cabeca}{
    \begin{tikzpicture}
        \node(Logo) {\includegraphics[width=2.5cm]{logo.png}};
        % \node(Logo) {\includegraphics[width=4.1cm]{logo.png}};

        \node(Local) at (Logo.north east) [anchor=north west, yshift=-0.25cm,
            align=center, execute at begin node=\setlength{\baselineskip}{3ex}]
            {
                \huge{\textbf{Universidade Federal do Amazonas}} \\
                \large{\textbf{Instituto de Ciências Exatas e Tecnologia}} \\
                \large{\textbf{\Description}}
            };

        \node(Ident) at (Local.south west) [anchor=north west, yshift=-0.25cm,
            align=left, execute at begin node=\setlength{\baselineskip}{2em}]
            {
                Professor: {\fontfamily{augie}\selectfont \Professor} \\
                Aluno:
            };
        % \draw [thick] (Logo.south west) -- ($(Logo.south west -| Local.south east)$);
        % \draw [red] (Logo.north west) rectangle (Logo.south east);
        % \draw [blue] (Local.north west) rectangle (Local.south east);
        % \draw [green] (Ident.north west) rectangle (Ident.south east);
    \end{tikzpicture}
}

\usepackage{fancyhdr}
\fancyhead{}
\fancyfoot{}
\fancyhead[c]{\cabeca}
\fancyfoot[r]{\fontfamily{augie}\selectfont Boa sorte!}


\pagestyle{fancy}
\renewcommand{\headrulewidth}{0pt}
\renewcommand{\footrulewidth}{0pt}

\newcommand{\ratio}[1]{(#1\% da nota)}
%-----------------------------------CUT HERE-----------------------------------

\def\Description{Física Geral II -- Prova 1 26/01/2023}
\def\Professor{Rodrigo de Farias Gomes}

\renewcommand{\vec}[1]{\overrightarrow{#1}}

\usepackage{siunitx}
\sisetup{locale = FR}

\begin{document}

\begin{tcolorbox}[colback=black!10, colframe=black!50, title=Observações]
    \begin{itemize}
        \item Todas as páginas com resposta devem ter o nome e matrícula do aluno
            escritos com caneta no início (cabeçalho) ou no final (rodapé). Páginas
            que não obedeçam a esse critério não serão usadas na avaliação
        \item As respostas podem ser escritas com lápis desde que legível
    \end{itemize}
\end{tcolorbox}

\begin{enumerate}
    \item \ratio{25} Um líquido, de massa específica \(\SI{900}{kg/m^3}\),
        escoa em um tubo horizontal com uma seção reta de \(\SI{1.90e-2}{m^2}\)
        na região \(A\) e uma seção reta de \(\SI{9.50e-2}{m^2}\) na região \(B\).
        A diferença de pressão entre as duas regiões é \SI{7.20e3}{Pa}.
        (a) Qual é a vazão e (b) qual é a vazão mássica?

    \item \ratio{25} Seja um tubo em forma de U modificado: o lado direito é mais curto que o lado
        esquerdo. A extremidade do lado direito está \(d=\SI{10,0}{cm}\) acima da bancada do laboratório.
        O raio do tubo é \SI{1,50}{cm}. Despeja-se água, lentamente, no lado esquerdo até que comece a
        transbordar do lado direito. Em seguida, um líquido, de massa específica \(\SI{0,80}{g/cm^3}\), é despejado
        lentamente no lado esquerdo até que a altura do líquido nesse lado seja de \SI{8,0}{cm} (o líquido não
        se mistura com a água). Que quantidade de água transborda do lado direito?

    \item \ratio{50} Um tanque cilíndrico de grande diâmetro está cheio d'água até uma profundidade
        \(D=\SI{0.50}{m}\). Um furo de seção reta \(A=\SI{6.5}{cm^2}\) no fundo do tanque permite 
        a drenagem de água. (a) Qual á a velocidade de escoamento da água, em metros cúbicos por segundo?
        (b) A que distância abaixo do fundo do tanque a seção reta do jorro é igual a um terço da área do furo?
\end{enumerate}

\end{document}
